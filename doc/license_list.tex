\documentclass[a4paper,10pt,oneside]{jsbook}
%
\usepackage{amsmath,amssymb,bm}
\usepackage{bm}
\usepackage[dvipdfmx]{graphicx}
\usepackage{ascmac}
\usepackage{makeidx}
\usepackage{txfonts}
\usepackage{indentfirst}
\usepackage{booktabs}
\usepackage{tabularx}
\usepackage{comment}
\AtBeginDvi{\special {pdf:tounicode EUC-UCS2}}
\usepackage[dvipdfmx, setpagesize=false, bookmarks=true, bookmarksnumbered=true]{hyperref}
%
\makeindex
%
\newcommand{\diff}{\mathrm{d}}            %微分記号
\newcommand{\divergence}{\mathrm{div}\,}  %ダイバージェンス
\newcommand{\grad}{\mathrm{grad}\,}       %グラディエント
\newcommand{\rot}{\mathrm{rot}\,}         %ローテーション
%
\setlength{\textwidth}{\fullwidth}
\setlength{\textheight}{44\baselineskip}
\addtolength{\textheight}{\topskip}
\setlength{\voffset}{-0.6in}
%

\begin{document}

%%%%%%%%%%%%%%%%%%%%%%%%%%%%%%%%%%%%%%%%%%%%%%%%%%%%%
% 表紙
\begin{titlepage}
\noindent
独立行政法人 理化学研究所 御中
\begin{center}
	\vspace{8cm}
	{\Huge \textbf{設計解探査システムのプロトタイプ整備} } \\
	\vspace{1cm}
	{\Huge \textbf{利用外部ライブラリ リスト}} \\
	\vspace{10cm}
	{\Large \textbf{2015年3月20日}} \\
	\vspace{0.5cm}
	{\Large \textbf{株式会社イマジカデジタルスケープ}}
\end{center}
\end{titlepage}

%%%%%%%%%%%%%%%%%%%%%%%%%%%%%%%%%%%%%%%%%%%%%%%%%%%%%
% 目次
\tableofcontents

%%%%%%%%%%%%%%%%%%%%%%%%%%%%%%%%%%%%%%%%%%%%%%%%%%%%%
% 本文
%%%%%%%%%%%%%%%%%%%%%%%%%%%%%%%%%%%%%%%%%%%%%%%%%%%%%
\chapter{使用モジュールおよび権利処理}
設計解探査システムのプロトタイプ整備において,利用した既存モジュールとその権利処理について記載する.
各利用モジュールの名称,権利者,権利処理内容,使用用途を表\ref{ops}に記載する.

\begin{table}[htbp]
\begin{center}
\caption{利用モジュールおよび権利処理一覧}
\label{ops}
\begin{tabular}{|l|l|l|l|}
\hline
名称 & 権利者 & 権利処理  & URL \\
\hline
\hline
jquery-1.11.2 & The jQuery Foundation. &  jQuery License & http://jquery.com/  \\
\hline
Handsontable & Nextgen & MIT License &  http://handsontable.com/ \\
\hline
jQuery Treeview & Jorn Zaefferer & MIT License & http://bassistance.de/jquery-plugins/jquery-plugin-treeview/\\
\hline
Qatrix 1.1 & Angel Lai &  MIT License & http://qatrix.com/\\
\hline
\end{tabular}
\end{center}
\end{table}

\end{document}
